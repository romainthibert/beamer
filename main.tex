\documentclass[aspectratio=169]{beamer}
\usepackage[T1]{fontenc}
\usepackage[french]{babel}

\usetheme{lurpa}

% Variables for title and closing slide
\title{Beamer template for scientific presentation}
\subtitle{Subtitle of the presentation or conference name}
\author[R.Thibert]{Romain Thibert}
\date{\today}
\institute[\url{https://lurpa.ens-paris-saclay.fr}]{4 Avenue des Sciences, 91190 Gif-sur-Yvette, France}

% Variable for ToC slides between sections
\summarytitle{Summary title for recap slides}

\begin{document}

\usebeamertemplate{titlepage}

\section{La première section}

\subsection{Première subsec}

\begin{frame}{Bonjour Titre}
Le corps d'une frame
\end{frame}

\subsection{Deuxième subsection}

\section{Deuxième section}

\begin{frame}{Deuxième Titre un peu plus long que le premier}
\begin{itemize}
    \item premier niveau
    \begin{itemize}
        \item niveau 2
        \begin{itemize}
            \item 3eme niveau
        \end{itemize}
    \end{itemize}
\end{itemize}
\end{frame}

\section{Autre section}

\subsection{Blocks}

\begin{frame}{Exemples de blocks}
    \begin{block}{Ceci est un block classique}
        Texte du block
    \end{block}
    \begin{block}{}
        Block sans titre
    \end{block}
    \begin{alertblock}{Alerted block}
        Corps du alerted block
    \end{alertblock}
\end{frame}

\subsection{Columns}

\begin{frame}{Slide a 2 colonnes}
    Texte avant les colonnes
    \begin{columns}
        \column{.6\textwidth}
            Première colonne
            \begin{block}{Bloc de la première colonne}
                Texte
            \end{block}
        \column{.4\textwidth}
            Deuxième colonne
            \begin{block}{Bloc de la deuxième colonne}
                Texte
            \end{block}
    \end{columns}
\end{frame}



\usebeamertemplate{endpage}

\end{document}
